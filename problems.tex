\section{Задачи}
\subsection{Задача 1}
Предприятие через 6 лет желает иметь на счете 1600 тыс. руб. Для этого оно должно делать ежегодный взнос в банк по схеме пренумерандо. Определить размер ежегодного взноса, если банк предлагает 13\% годовых (проценты сложные).

\begin{center}
	Решение
\end{center}

По формуле наращения аннуитета имеем:
\[ FV(A) = A\dfrac{(1+E)^n - 1}{E} \times (1 + E) ;\]
\[ A = \dfrac{FV(A)}{\dfrac{(1+E)^n - 1}{E} \times (1 + E)} ;\]
\[ A = \dfrac{1600}{\dfrac{(1+0,13)^6 - 1}{0,13} \times (1 + 0,13)}; \]
\[ A = 170,13  \]

Ответ: размер ежегодного взноса по схеме пренумерандо составляет 170,13 тыс. руб.

\subsection{Задача 2}
ОАО «Петя + Миша» имеет возможность профинансировать инвестиционный проект на 75\%  за счет заемного капитала и на 25\% за счет собственных средств. Средняя процентная  ставка за кредит составляет 11\%, цена собственного капитала 6\%. Доходность проекта планируется на уровне 15\%. Следует ли реализовать данный инвестиционный проект?

\begin{center}
	Решение
\end{center}

Доходность инвестиционного проекта должна превышать стоимость капитала. Для сравнения этих показателей, рассчитаем средневзвешенную стоимость капитала (WACC) и сравним полученное значение с уровнем доходности.
\[ WACC = 0,75 \times 11\% + 0,25 \times 6\%;\]
\[ WACC = 9,75\%.\]

Ответ: полученное значение средневзвешенной стоимости капитала 9,75\% ниже доходности 15\%, следовательно данный инвестиционный проект является целесообразным.


\subsection{Задача 3}
По проекту стоимость годового выпуска продукции будущим предприятием должна составлять 100 млн. рублей, а затраты на 1 рубль товарной продукции --- 0,85 рублей. Проектный срок строительства объекта составляет 3 года. Определите проектную эффективность капитальных вложений, срок их окупаемости и как изменятся показатели эффективности капитальных вложений при условии сокращения срока  строительства на 3 месяца. Сметная стоимость строительства объекта 50 млн. руб.

\begin{center}
	Решение
\end{center}

Эффективность капитальных вложений --- соотношение между затратами на производство основных фондов и получаемыми результатами. Из условия вычислим проектную рентабельность предприятия:
\[ R = \frac{\text{Выручка}}{\text{Затраты}}; \]
\[ R = \frac{\text{100}}{\text{85}} = 1,1765; \]

Рентабельность предприятия составит 17,65\%.

Годовой доход (P) составляет: 100 -- 85 = 15 млн. руб.

Эффективность капитальных вложений:
\[ROI = \frac{P}{IC} ;\]
\[ROI = \frac{15}{50} = 0,3 \]
Срок окупаемости (T):
\[ T = \frac{IC}{P}; \]
\[ T = \frac{50}{15} = 3,33\  \text{г.} \]

\subsection{Задача 4}
На основании исходных данных таблицы и вариантов ставки дисконта рассчитать  показатели эффективности: NPV; PI; DPP и сделать выводы о влиянии динамики денежного потока на показатели эффективности проектов. Ставка дисконта 30.

% Please add the following required packages to your document preamble:
% \usepackage{graphicx}
\begin{table}[!h]
	\small
	\caption{Характеристика инвестиционных проектов, млн.руб.}
	\label{my-label}
	\begin{tabularx}{\textwidth}{|k{3cm}|K{3cm}|K{3cm}|K{3cm}|}
			\hline
			Годы & Проект А & Проект Б & Проект В \\ \hline
			0    & -250     & -250     & -250     \\ \hline
			1    & 50       & 200      & 125      \\ \hline
			2    & 100      & 150      & 125      \\ \hline
			3    & 150      & 100      & 125      \\ \hline
			4    & 200      & 50       & 125      \\ \hline
		\end{tabularx}
\end{table}

\begin{center}
	Решение
\end{center}






















\subsection{Задача 5}
На приобретение мясоперерабатывающим предприятием новой технологической линией израсходовано 39 млн.руб.

Срок службы оборудования --- 5 лет. Амортизация начисляется линейным методом. Чистая прибыль за расчетный период прогнозируется по годам: 1-й --- 10 млн. руб., 2-й --- 12 млн. руб., 3-й --- 10 млн. руб., 4-й --- 10 млн. руб., 5-й --- 9 млн. руб. Цена капитала, инвестируемого в проект --- 25\%. Определите целесообразность приобретения новой  технологической линии на основе расчета внутренней нормы доходности проекта. Построить график зависимости NPV от ставки дисконта Е.