\section{Задачи}
\subsection{Задача 1}
Предприятие через 6 лет желает иметь на счете 1600 тыс. руб. Для этого оно должно делать ежегодный взнос в банк по схеме пренумерандо. Определить размер ежегодного взноса, если банк предлагает 13\% годовых (проценты сложные).

\begin{center}
	Решение
\end{center}

По формуле наращения аннуитета имеем:
\[ FV(A) = A\dfrac{(1+E)^n - 1}{E} \times (1 + E) ;\]
\[ A = \dfrac{FV(A)}{\dfrac{(1+E)^n - 1}{E} \times (1 + E)} ;\]
\[ A = \dfrac{1600}{\dfrac{(1+0,13)^6 - 1}{0,13} \times (1 + 0,13)}; \]
\[ A = 170,13  \]

Ответ: размер ежегодного взноса по схеме пренумерандо составляет 170,13 тыс. руб.

\subsection{Задача 2}
ОАО «Петя + Миша» имеет возможность профинансировать инвестиционный проект на 75\%  за счет заемного капитала и на 25\% за счет собственных средств. Средняя процентная  ставка за кредит составляет 11\%, цена собственного капитала 6\%. Доходность проекта планируется на уровне 15\%. Следует ли реализовать данный инвестиционный проект?

\begin{center}
	Решение
\end{center}

Доходность инвестиционного проекта должна превышать стоимость капитала. Для сравнения этих показателей, рассчитаем средневзвешенную стоимость капитала (WACC) и сравним полученное значение с уровнем доходности.
\[ WACC = 0,75 \times 11\% + 0,25 \times 6\%;\]
\[ WACC = 9,75\%.\]

Ответ: полученное значение средневзвешенной стоимости капитала 9,75\% ниже доходности 15\%, следовательно данный инвестиционный проект является целесообразным.


\subsection{Задача 3}
По проекту стоимость годового выпуска продукции будущим предприятием должна составлять 100 млн. рублей, а затраты на 1 рубль товарной продукции --- 0,85 рублей. Проектный срок строительства объекта составляет 3 года. Определите проектную эффективность капитальных вложений, срок их окупаемости и как изменятся показатели эффективности капитальных вложений при условии сокращения срока  строительства на 3 месяца. Сметная стоимость строительства объекта 50 млн. руб.

\begin{center}
	Решение
\end{center}

Эффективность капитальных вложений --- соотношение между затратами на производство основных фондов и получаемыми результатами. Из условия вычислим проектную рентабельность предприятия:
\[ R = \frac{\text{Выручка}}{\text{Затраты}}; \]
\[ R = \frac{\text{100}}{\text{85}} = 1,1765; \]

Рентабельность предприятия составит 17,65\%.

Годовой доход (P) составляет: 100 -- 85 = 15 млн. руб.

Эффективность капитальных вложений:
\[ROI = \frac{P}{IC} ;\]
\[ROI = \frac{15}{50} = 0,3 \]
Срок окупаемости (T):
\[ T = \frac{IC}{P}; \]
\[ T = \frac{50}{15} = 3,33\  \text{г.} \]

\subsection{Задача 4}
На основании исходных данных таблицы и вариантов ставки дисконта рассчитать  показатели эффективности: NPV; PI; DPP и сделать выводы о влиянии динамики денежного потока на показатели эффективности проектов. Ставка дисконта 30.

% Please add the following required packages to your document preamble:
% \usepackage{graphicx}
\begin{table}[!h]
	\small
	\caption{Характеристика инвестиционных проектов, млн.руб.}
	\label{my-label}
	\begin{tabularx}{\textwidth}{|K{3.25cm}|K{4cm}|K{4cm}|K{4cm}|}
			\hline
			Годы & Проект А & Проект Б & Проект В \\ \hline
			0    & -250     & -250     & -250     \\ \hline
			1    & 50       & 200      & 125      \\ \hline
			2    & 100      & 150      & 125      \\ \hline
			3    & 150      & 100      & 125      \\ \hline
			4    & 200      & 50       & 125      \\ \hline
		\end{tabularx}
\end{table}

\begin{center}
	Решение
\end{center}

Рассчитаем показатели эффективности для каждого проекта (см. таблицы 2--4).
	
На основании полученных показателей можно сделать следующие выводы. При ставке дисконта (30\%) наибольшую эффективность имеет проект Б с наиболее быстрым сроком окупаемости 2,16 года и и индексом доходности 1,223. Проект В имеет больший срок окупаемости 3,52 года и индекс доходности 1,084. Проект А оказался неэффективным, так как первоначальные инвестиции оказались больше дисконтированных денежных потоков от реализации проекта.

Таким образом, наибольшую эффективность показал проект, в процессе реализации которого вложенные средства возвращаются быстрее.
%Рассчитаем NPV для проета А. Для этого найдем дисконтированный денежный поток.\\
%\[50 \cdot \dfrac{1}{(1+0,3)^1} = 38,5\]
%\[100 \cdot \dfrac{1}{(1+0,3)^2} = 59,2\]
%\[150 \cdot \dfrac{1}{(1+0,3)^3} = 68,3\]
%\[200 \cdot \dfrac{1}{(1+0,3)^4} = 70\]
%\[NPV = -250+38.5+59.2+68.3+70=-14\]
%
%Рассчитаем PI.
%\[PI = \dfrac{-14}{250}+1=-1.056\]

%Рассчитаем DPP.
%0 год: -250\\
%1 год: -211,5\\
%2 год: -152,3\\
%3 год: -84\\
%4 год: -14.

\begin{table}[!h]
	\caption{проект А}
	\label{project_A}
	\small
	\setlength{\extrarowheight}{1.2mm}
		\begin{tabularx}{\textwidth}{|K{1.12cm}|K{1.12cm}|p{8cm}|p{5cm}|}
		\hline
		&& \multicolumn{1}{c|}{$NPV$}                      & \multicolumn{1}{c|}{$DPP$} \\ \hline
		0 &$ -250$&                                                                   &  $    -250   $                 \\ \hline
		1 &$5$0& $50 \cdot \frac{1}{(1+0,3)^1} = 38,5$  &$ -250 +38,5   = -211,5     $            \\ \hline
		2 & $100$&$100 \cdot \frac{1}{(1+0,3)^2} = 59,2$ & $-211,5  +59,2=152,3      $           \\ \hline
		3 &$150 $&$150 \cdot \frac{1}{(1+0,3)^3} = 68,3$ &$-152,3  +68,3=-84    $             \\ \hline
		4 &$200$& $200 \cdot \frac{1}{(1+0,3)^4} = 70$  & $-84 +70= -14     $                \\ \hline
		&&$NPV = -250+38,+59,2+68,3+70=-14$     & $-14 $                     \\ \hline
		&&$PI = \frac{-14}{250}+1=-1,056$                                        & $DPP = $                 \\ \hline
		\end{tabularx}
		\end{table}

\begin{table}[!h]
	\label{project_B}
	\caption{проект Б}
	\small
	\setlength{\extrarowheight}{1.2mm}
	\begin{tabularx}{\textwidth}{|K{1.12cm}|K{1.12cm}|p{8cm}|p{5cm}|}
		\hline
		& &\multicolumn{1}{c|}{$NPV$}                      & \multicolumn{1}{c|}{$DPP$} \\ \hline
		0 & $-250$&                                                                   &  $    -250   $                 \\ \hline
		1 & $200$&$200 \cdot \frac{1}{(1+0,3)^1} = 153,9$  &$ -250 +153,9   = -96,1     $            \\ \hline
		2 & $150$&$150 \cdot \frac{1}{(1+0,3)^2} = 88,8$ & $-96,1  +88,8=-7,3      $           \\ \hline
		3 &$100$& $100 \cdot \frac{1}{(1+0,3)^3} = 45,5$ &$-7,3  +45,5=38,2    $             \\ \hline
		4 &$50$&$50 \cdot \frac{1}{(1+0,3)^4} =17,5$  & $38,2 +17,5= 55,7     $                \\ \hline
		&&$NPV = -250+38,+59,2+68,3+70=55,7$     & $55,7 $                     \\ \hline
		&&$PI = \frac{55,7}{250}+1=1,223$                                        & $DPP = 2,\frac{7,3}{45,5}=2,16$                 \\ \hline
	\end{tabularx}
\end{table}

\begin{table}[!h]
	\caption{проект В}
	\label{project_C}
	\small
	\setlength{\extrarowheight}{1.2mm}
	\begin{tabularx}{\textwidth}{|K{1.12cm}|K{1.12cm}|p{8cm}|p{5cm}|}
		\hline
		& &\multicolumn{1}{c|}{$NPV$}                    & \multicolumn{1}{c|}{$DPP$} \\ \hline
		0 &$-250$&                                                                     &  $    -250   $                 \\ \hline
		1 & $125$&$125 \cdot \frac{1}{(1+0,3)^1} = 96,2$ &$ -250 +96,2   =-153,8     $            \\ \hline
		2 & $125$&$125 \cdot \frac{1}{(1+0,3)^2} = 74$    & $-153,8  +74=-79,8      $           \\ \hline
		3 &$125$& $125 \cdot \frac{1}{(1+0,3)^3} = 56,9$ &$-79,8  +56,9=-22,9    $             \\ \hline
		4 & $125$&$125 \cdot \frac{1}{(1+0,3)^4} =43,8$  & $-22,9 +43,8= 20,9     $                \\ \hline
		&&$NPV = -250+96,2+74+56,9+43,8=20,9$     & $20,9 $                     \\ \hline
		&&$PI = \frac{20,9}{250}+1=1,084$               & $DPP = 3,\frac{22,9}{43,8}=3,52$                 \\ \hline
	\end{tabularx}
\end{table}








\subsection{Задача 5}
На приобретение мясоперерабатывающим предприятием новой технологической линией израсходовано 39 млн.руб.

Срок службы оборудования --- 5 лет. Амортизация начисляется линейным методом. Чистая прибыль за расчетный период прогнозируется по годам: 1-й --- 10 млн. руб., 2-й --- 12 млн. руб., 3-й --- 10 млн. руб., 4-й --- 10 млн. руб., 5-й --- 9 млн. руб. Цена капитала, инвестируемого в проект --- 25\%. Определите целесообразность приобретения новой  технологической линии на основе расчета внутренней нормы доходности проекта. Построить график зависимости NPV от ставки дисконта Е.